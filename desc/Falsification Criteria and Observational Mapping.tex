% ============================================================
\section*{Falsification Criteria and Observational Mapping}
\addcontentsline{toc}{section}{Falsification Criteria and Observational Mapping}

% ============================================================
\section{Falsification Criteria}

A theory must specify conditions under which it can be rejected.
The framework presented here is falsifiable under the following criteria.

\subsection{Absence of Matter Condensation in High Wave-Activity Zones}

\textbf{Claim:}
High-intensity wave regions are capable of producing matter
via condensation when local energy thresholds are exceeded.

\textbf{Falsification condition:}
If observational surveys demonstrate that regions with:
\begin{itemize}
\item extreme electromagnetic or gravitational wave activity,
\item minimal pre-existing baryonic matter,
\item long-term wave coherence,
\end{itemize}
show \textbf{no statistically significant excess of matter}
beyond what is predicted by transport, accretion, or known decay processes,
then the condensation hypothesis is invalid.

This includes:
\begin{itemize}
\item absence of unexplained gas clouds,
\item absence of anomalous particle populations,
\item absence of matter without progenitor signatures.
\end{itemize}

\subsection{Strict Non-Interaction of Distinct Radiation Types}

\textbf{Claim:}
Radiation types may share a common wave substrate or weak coupling
prior to interaction with matter.

\textbf{Falsification condition:}
If controlled laboratory or astrophysical measurements establish
that radiation of distinct spectral classes:
\begin{itemize}
\item cannot exchange energy,
\item cannot interfere even partially,
\item cannot produce non-linear effects under any phase configuration,
\end{itemize}
then the universal wave substrate interpretation is ruled out.

In such a case, radiation must be treated as fully independent instantiations,
and the simplified wave-energy framework collapses.

\subsection{No Energy Deficit at Phase Cancellation}

\textbf{Claim:}
Wave anti-phase cancellation may result in energy redistribution or loss
not accounted for by pressure or radiation emission.

\textbf{Falsification condition:}
If precise experiments show that:
\begin{itemize}
\item all input energy is fully recoverable as heat, radiation, or mechanical work,
\item no residual or missing energy is detected at cancellation points,
\end{itemize}
then transported wave energy cannot exceed classical descriptions,
invalidating the proposed energy upper-bound measurement method.

\subsection{Gravitation Fully Reducible to Curvature Alone}

\textbf{Claim:}
Gravitation arises from gradients of field allowance rather than pure spacetime curvature.

\textbf{Falsification condition:}
If all gravitational phenomena—including:
\begin{itemize}
\item galaxy rotation curves,
\item cluster dynamics,
\item strong-field lensing,
\end{itemize}
are fully and consistently explained by curvature-only models
without invoking hidden energy reservoirs or field gradients,
then the work-density interpretation of gravitation is unnecessary and rejected.

\subsection{Persistence of Stable Universes Without External Feeding}

\textbf{Claim:}
Long-term stability of matter and structure requires continuous or residual external feeding.

\textbf{Falsification condition:}
If evidence is found that:
\begin{itemize}
\item universes remain structurally stable,
\item maintain star formation indefinitely,
\item and avoid terminal synthesis collapse,
\end{itemize}
without any detectable external energy contribution,
then the framework feeding requirement is false.

% ============================================================
\section{Mapping to Observables}

\subsection{Astrophysical Observables}

\begin{center}
\begin{tabular}{|l|l|}
\hline
\textbf{Framework Prediction} & \textbf{Observable Signature} \\
\hline
Wave-driven condensation & Gas clouds without progenitors \\
Node exhaustion & Old stellar populations, no star formation \\
Terminal synthesis & Overabundance of heavy elements \\
Field decay & Redshift-like energy loss without expansion \\
Work density gradients & Non-Newtonian gravitational behavior \\
\hline
\end{tabular}
\end{center}

\subsection{Galactic Core Diagnostics}

Predicted properties of galactic centers:
\begin{itemize}
\item dominance of old, metal-rich stars,
\item absence of young stellar populations,
\item limited or extinguished accretion activity,
\item possible absence of a classical black hole.
\end{itemize}

These signatures indicate exhaustion of external feeding
and transition into a terminal work regime.

\subsection{Intergalactic Medium}

The framework predicts that the intergalactic medium:
\begin{itemize}
\item contains matter not fully attributable to galactic outflows,
\item correlates with historical wave activity,
\item exhibits chemical compositions inconsistent with transport alone.
\end{itemize}

\subsection{Laboratory Observables}

Potential measurable effects:
\begin{itemize}
\item missing energy at wave interference nodes,
\item non-thermal particle emergence,
\item energy imbalance beyond radiation pressure.
\end{itemize}

Failure to detect these effects under sufficient experimental sensitivity
directly falsifies the wave-energy component of the theory.

% ============================================================
\section{Interpretational Boundary}

The framework does not assume intent, agency, or purpose.
It asserts only that:
\begin{quote}
The observed universe behaves as a system operating outside its nominal regime.
\end{quote}

This statement remains valid or invalid solely on empirical grounds,
independent of metaphysical interpretation.
