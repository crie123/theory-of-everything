\section{Introduction}

Current cosmological and physical frameworks rely heavily on fragmented descriptions: general relativity treats gravity geometrically, quantum field theory treats particles as excitation states, and cosmology adds parameters such as dark energy and dark matter to match observations.

Yet these models fail to answer key fundamental questions:
\begin{itemize}
    \item Why does mass exist as a property? Why does it vanish in some particles?
    \item What is gravity really? Is it fundamental or emergent?
    \item Why does time behave differently in different locations?
    \item Why do isotopic stabilities follow non-random patterns?
    \item Why do quantum tunneling and field thresholds act as if there's "hidden work" being done?
\end{itemize}

This paper introduces the Work-Time-Field (WTF) theory --- a unified construct where mass, time, decay, and gravity are all consequences of a single principle:

\[
E = t \cdot g
\]

Here, \( E \) is energy as executed work, \( t \) is the temporal moment (when), and \( g \) is the field-location vector (where). The theory is built on the observation that observable energy, interaction, and existence arise only when work is performed in the field.

Rather than assuming mass as an intrinsic property, WTF suggests that particles emerge and retain stability only through sustained localized work. Decay, tunneling, field collapse, and gravitational curvature are all interpreted as breakdowns or redistributions of this field work.

The framework further demonstrates its reach by:
\begin{itemize}
    \item Predicting realistic isotope distributions, including the anomalous stability of Tin;
    \item Explaining quantum tunneling as trans-field leakage without needing imaginary energy;
    \item Modeling cosmological expansion not as outward movement, but as field exhaustion (a “collapsing scaffolding”);
    \item Providing clear predictions for experimental verification using memory degradation, tunneling, isotope spread, and biological slowdown under field modulation.
\end{itemize}
