\section{Theoretical Foundation}

The Work-Time-Field (WTF) model is based on a single postulate:
\begin{center}
    \textit{All observable phenomena are the result of localized work in a directional field, with time and field-position as primary variables.}
\end{center}

We define:

\begin{itemize}
    \item \( E \) — observable energy, defined as executed work (not potential);
    \item \( t \) — temporal moment (not duration), the "when" of work;
    \item \( g \) — gravitational or field-location vector, representing "where" the work is performed.
\end{itemize}

The core relation becomes:

\[
E = t \cdot g
\]

This is not a classical scalar product, but a phase-linked operation: "when" and "where" together define whether work manifests as observable energy.

\subsection{Mass as Function of Work}

Rather than treating mass as an intrinsic property, we redefine it as:

\[
m = \frac{E}{c^2} = \frac{t \cdot g}{c^2}
\]

Thus, mass is emergent — the result of localized field work at a specific time and place.

\subsection{Stability and Decay}

A stable particle is one whose temporal-field balance remains resonant.  
When the work phase diverges — due to external excitation, spatial shift, or loss of coherence — decay occurs.

\subsection{Tunneling as Phase Leakage}

Quantum tunneling is interpreted not as a probability, but as **phase leakage** across a barrier:

\[
P_{tunnel} \propto \exp\left(-\frac{W}{t \cdot g_{\text{leak}}}\right)
\]

Where \( g_{\text{leak}} \) is the effective slope of the field beyond the barrier. This explains why tunneling probability can remain non-zero without imaginary energy.

\subsection{Gravitation and Collapse}

Gravitational curvature is the redistribution of ongoing work density.  
Black holes are regions where the localized energy flux exceeds the structural holding capacity — causing collapse.

The universe is not expanding — it is **depleting**.  
The edge is not a growing boundary, but a zone of **exhausted field topology**, beyond which work no longer manifests.
