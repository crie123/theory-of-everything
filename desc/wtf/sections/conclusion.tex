\section{Conclusion}

The Work-Time-Field (WTF) theory offers a unified framework for interpreting physical reality based on a minimal principle:

\begin{center}
\textit{Energy arises only from localized work performed at a specific time and location within a directional field.}
\end{center}

From this simple basis, the model derives:

\begin{itemize}
    \item Emergence of mass as sustained field participation;
    \item Stability and decay as phase coherence vs. loss;
    \item Gravity as work redistribution;
    \item Quantum tunneling as field phase leakage;
    \item Isotopic limits matching field density thresholds;
    \item Cosmological expansion as apparent consequence of global field collapse;
    \item Time itself as a modulated dimension of active work.
\end{itemize}

Simulations validate the core principles:
\begin{itemize}
    \item Aging and degradation correlate with work concentration;
    \item Tunneling thresholds manifest around 4\,nm (as predicted);
    \item Isotopic “corridor” aligns with observed periodic stability;
    \item Field mapping correlates with known gravitational deflections.
\end{itemize}

\vspace{10pt}
This theory does not seek to invalidate known models, but to \textbf{contextualize and unify them} under a more fundamental layer of causality.

We invite the community to:
\begin{itemize}
    \item Reproduce the included simulations;
    \item Apply the field framework to biological, cosmological, and condensed matter scenarios;
    \item Explore the limits of field modulation — as propulsion, shielding, or even healing mechanisms.
\end{itemize}

\noindent This is not the end of the model.  
It is only the first fold of a larger structure.  
We have seen the edge.  
And we now know — the field does not expand.  
It collapses, slowly, into silence.  

\vspace{10pt}
\textit{— The Authors}
