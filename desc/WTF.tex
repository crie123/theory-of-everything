\documentclass[12pt,a4paper]{article}

\usepackage[T1]{fontenc}
\usepackage[utf8]{inputenc}
\usepackage[russian,english]{babel}
\usepackage{amsmath,amssymb}
\usepackage{geometry}
\usepackage{graphicx}
\usepackage{hyperref}
\usepackage{physics}
\usepackage{bm}

\geometry{margin=2.5cm}

\title{Unified Work--Time--Field Framework\\
\large A Carcass-Based Model of Matter, Energy, and Stability}
\author{Kirill Nikitenko}
\date{2025}

\begin{document}
\maketitle

\begin{abstract}
This document presents a unified and structured formulation of the Work--Time--Field (WTF)
framework. The theory interprets matter, radiation, and spacetime phenomena as manifestations
of structured work performed within a directional field geometry (the ``carcass'').
The framework integrates gravitational interaction, isotopic stability, neutrino behavior,
wave--particle duality, and limits of reversibility into a single conceptual and mathematical
model. Experimental, observational, and engineering implications are discussed.
\end{abstract}

\tableofcontents
\newpage

% ============================================================
\section{Foundational Axiom}

\subsection{Energy as Structured Work}

The central axiom of the framework states:

\begin{equation}
E = t \cdot g
\end{equation}

where:
\begin{itemize}
\item $E$ --- energy interpreted as structured work,
\item $t$ --- time during which the work is performed,
\item $g$ --- a coordinate of the carcass field, interpreted as local allowance or graviton
participation density.
\end{itemize}

Energy is not treated as an abstract scalar, but as a process localized in both time and
field geometry.

\subsection{Graviton as Field Coordinate}

A ``graviton'' in this framework is not a particle but a coordinate state of the carcass
field at which work can occur. The number of participating gravitons $N_g$ determines how
energy is distributed and stabilized.

% ============================================================
\section{Field Architecture and Ontology}

\subsection{The Carcass Field}

The Universe is modeled as a unified field containing a directional scaffold (carcass)
that defines allowed paths for work and wave propagation.

\begin{itemize}
\item Where the carcass exists, waves, mass, and interaction are possible.
\item Where the carcass is absent, no work, mass, or interaction can occur.
\end{itemize}

Dark matter is interpreted as carcass geometry outside the observable interaction spectrum.

\subsection{Wave and Particle as Field States}

\begin{itemize}
\item \textbf{Wave}: a distributed gradient of work along the carcass.
\item \textbf{Particle}: a localized, stabilized resonance of work.
\end{itemize}

Wave--particle duality arises from localization versus distribution along the same field
structure.

% ============================================================
\section{Matter Stability and Energy Bands}

\subsection{Stability Band}

Stable matter exists within an intermediate allowance region of the carcass:

\begin{itemize}
\item Upper branch: quarks, Higgs decay products, photons.
\item Equilibrium: electrons, stable nuclei.
\item Lower branch: neutrinos, decay remnants.
\end{itemize}

\subsection{Isotopic Stability Criterion}

A nucleus is stable when the neutron-to-proton ratio aligns with the local field configuration:

\begin{equation}
\frac{N_n}{N_p} \in [1.0,\,1.6)
\end{equation}

Stability is relative and depends on local carcass allowance, not solely on particle count.

\subsection{Neutron as Energy Reservoir}

The free neutron functions as an energy buffer. Its decay:

\begin{equation}
n \rightarrow p + e^- + \bar{\nu}_e \qquad (\Delta E = 0.782\ \text{MeV})
\end{equation}

signals the inability of the system to stabilize within the local allowance.

% ============================================================
\section{Work Distribution and Decay}

\subsection{Work per Graviton}

\begin{equation}
W = \frac{E}{N_g}
\end{equation}

Exceeding the local allowance per graviton leads to decay or phase transition.

\subsection{Decay Rate Modification}

External energy feeding modifies decay probability:

\begin{equation}
\Delta \lambda \approx \alpha \cdot \frac{P_{\text{in}}}{E_{\text{rest}}}
\end{equation}

where $\alpha$ is a sensitivity coefficient and $P_{\text{in}}$ is injected power.

Astrophysically significant effects require stellar-core or pre-supernova flux densities.

% ============================================================
\section{Wave Propagation and Interference}

\subsection{Wave as Field Displacement}

A wave is defined as a structured displacement of the carcass field. Propagation does not
consume energy for self-transport, except during interaction or interference.

\subsection{Phase Cancellation}

Anti-phase wave superposition leads to field tension cancellation, implying:

\begin{itemize}
\item measurable limits on transported wave energy,
\item potential local condensation effects at collision points,
\item a method for experimentally bounding wave energy independent of radiation pressure.
\end{itemize}

% ============================================================
\section{Gravitation and Geometry}

\subsection{Gravitation as Carcass Configuration}

Gravity does not act as a classical force but as a manifestation of carcass geometry.
The center of mass represents a convergence point of field allowance, not an interaction
with empty space.

\subsection{Gravitational Waves}

Amplitude depends on energy distribution:

\begin{equation}
A \sim f(E_\uparrow, E_\downarrow)
\end{equation}

Low observed amplitude during massive mergers implies energy transfer into lower
(field-closing) branches.

% ============================================================
\section{Warp and Displacement Mechanics}

\subsection{Jump Without Translation}

Displacement is achieved by restructuring the carcass:

\begin{equation}
F(x,y) = e^{-k(x^2 + y^2)}
\end{equation}

The object does not move through space but rebinds to a new coordinate without performing
classical work.

\subsection{Engineering Implications}

\begin{itemize}
\item No relativistic desynchronization if no work is performed.
\item Dual-charge stabilization prevents destructive phase shear.
\end{itemize}

% ============================================================
\section{Experimental Proposals}

\subsection{Laboratory}

\begin{itemize}
\item Data degradation under varying field density.
\item Decay-rate modification under external energy feed.
\item PN-junction closure without physical contact.
\end{itemize}

\subsection{Astronomical}

\begin{itemize}
\item Galaxy core classification by stellar age distribution.
\item Neutrino clustering in high-activity regions.
\item Absence or presence of matter formation near wave collision zones.
\end{itemize}

% ============================================================
\section{Limit of Reversibility}

The reverse transition from matter to coherent wave requires total phase synchronization:

\begin{equation}
\lim_{E \to E_{\text{crit}}} \frac{dW}{dt} = 0
\end{equation}

Such a state exceeds any achievable physical condition, defining a natural boundary of
matter.

\begin{itemize}
\item Radiation --- partial phase loss.
\item Black holes --- closure of phase space.
\item Neutrinos --- minimal coherent leakage.
\end{itemize}

% ============================================================
\section{Conclusion}

The WTF framework interprets stable matter as a terminal resonance of a universal work
field. Apparent eternity of particles arises naturally from field topology rather than
symmetry constraints.

Further progress lies not in deeper subdivision of matter, but in understanding the
topology, alignment, and engineering control of resonant work states.

\bigskip
\begin{center}
\emph{Beyond this point lies not substance, but structure.}
\end{center}

\end{document}
