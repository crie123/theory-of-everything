% ============================================================
\appendix
\section*{Appendix}
\addcontentsline{toc}{section}{Appendix}

% ============================================================
\section{Collected Formulae and Derived Relations}

\subsection{Fundamental Relations}

\begin{equation}
E = t \cdot g
\end{equation}

Energy is interpreted as structured work performed over time within the carcass field.

\begin{equation}
W = \frac{E}{N_g}
\end{equation}

Work per graviton determines stability, decay, or phase transition.

\begin{equation}
\rho_W = \frac{1}{V}\frac{dW}{dt}
\end{equation}

Local work density governs singularity formation, condensation, and structural collapse.

\subsection{Stability and Decay}

\begin{equation}
\frac{N_n}{N_p} \in [1.0,\,1.6)
\end{equation}

Relative nuclear stability condition depending on local field allowance.

\begin{equation}
n \rightarrow p + e^- + \bar{\nu}_e \quad (\Delta E = 0.782\ \text{MeV})
\end{equation}

Neutron decay interpreted as loss of local stabilization capacity.

\begin{equation}
\Delta \lambda \approx \alpha \cdot \frac{P_{\text{in}}}{E_{\text{rest}}}
\end{equation}

External energy feed modifies decay probability.

\subsection{Wave Transport}

\begin{equation}
\frac{dE_{\text{wave}}}{dx} \approx 0 \quad (\text{in vacuum, no interaction})
\end{equation}

Wave propagation conserves transported energy except at interaction events.

\begin{equation}
E_{\text{loss}} = E_1 + E_2 - E_{\text{res}}
\end{equation}

Energy balance at wave interference points.

% ============================================================
\section{Wave Energy Measurement and Phase Interference}

\subsection{Anti-Phase Cancellation}

If two coherent waves of equal amplitude and opposite phase intersect:

\begin{equation}
\psi_1 = A \sin(\omega t), \quad
\psi_2 = -A \sin(\omega t)
\end{equation}

then:

\begin{equation}
\psi_{\text{total}} = 0
\end{equation}

This implies:
\begin{itemize}
\item cancellation of field displacement,
\item non-radiative loss channel,
\item measurable upper and lower bounds on transported wave energy.
\end{itemize}

\subsection{Condensation Hypothesis}

At the collision point of high-energy counter-propagating waves:

\begin{equation}
E_{\text{local}} \ge E_{\text{cond}}
\end{equation}

where $E_{\text{cond}}$ is a threshold for particle formation.

Expected observables:
\begin{itemize}
\item excess particle density,
\item pair creation without external matter source,
\item localized entropy decrease followed by rapid thermalization.
\end{itemize}

% ============================================================
\section{Experimental Proposals}

\subsection{Laboratory Experiments}

\subsubsection{Wave Interference Energy Bounds}

\begin{itemize}
\item Generate two high-power coherent beams.
\item Introduce controlled phase shift.
\item Measure:
\begin{itemize}
  \item radiation pressure,
  \item thermal deposition,
  \item missing energy at cancellation.
\end{itemize}
\end{itemize}

Goal: determine transported energy independent of pressure effects.

\subsubsection{Cross-Spectrum Interference}

Attempt partial phase alignment between:
\begin{itemize}
\item microwave and infrared,
\item X-ray and gamma bands.
\end{itemize}

Observation of any non-linear interaction would indicate shared wave substrate.

\subsection{Astrophysical Observations}

\subsubsection{High-Activity Wave Zones}

Target regions:
\begin{itemize}
\item pulsar magnetospheres,
\item quasar jets,
\item AGN cores,
\item jet--medium interaction fronts.
\end{itemize}

Search for:
\begin{itemize}
\item matter density exceeding accretion models,
\item gas without progenitor signatures,
\item anomalous elemental ratios.
\end{itemize}

\subsubsection{Stellar Core Diagnostics}

If central galactic regions exhibit:
\begin{itemize}
\item old stellar populations,
\item absence of star formation,
\item high metallicity,
\end{itemize}

this implies exhaustion of external field feeding and terminal work regime.

% ============================================================
\section{Gravitation and Center-of-Mass Interpretation}

\subsection{Center of Mass as Field Convergence}

The center of mass is interpreted not as attraction to empty space but as convergence of
field allowance gradients:

\begin{equation}
\nabla g \neq 0 \quad \Rightarrow \quad a_{\text{grav}} \neq 0
\end{equation}

This explains:
\begin{itemize}
\item long-range action,
\item insensitivity to electromagnetic shielding,
\item non-linearity in multi-body systems.
\end{itemize}

\subsection{Implications for the Three-Body Problem}

Indeterminacy arises from overlapping field configurations rather than force superposition.

% ============================================================
\section{Limits of the Framework}

\subsection{Reversibility Boundary}

\begin{equation}
\lim_{\text{phase coherence} \to 1} E_{\text{matter}} \to E_{\text{wave}}
\end{equation}

This limit is physically unreachable under known cosmological conditions.

\subsection{Thermal Death Scenario}

In absence of external feeding:
\begin{itemize}
\item work nodes extinguish,
\item synthesis peaks briefly,
\item system collapses into inert field geometry.
\end{itemize}

Observed universe characteristics are consistent with a late-stage or degraded operating mode.

% ============================================================
\section{Concluding Remarks}

The presented framework remains experimentally falsifiable.
Its central claim is not metaphysical but operational:

\begin{quote}
Matter behaves as a stabilized record of performed work within a structured field.
\end{quote}

Further validation depends on controlled wave-interference experiments and
high-resolution astrophysical surveys of energy--matter balance anomalies.
